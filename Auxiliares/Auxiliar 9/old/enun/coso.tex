\hypertarget{circuitos-y-cachuxe9}{%
\section{Circuitos y Caché}\label{circuitos-y-cachuxe9}}

\hypertarget{circuitos}{%
\subsection{Circuitos}\label{circuitos}}

Hacer en logisim un circuito que genere los números de Fibonacci

\hypertarget{cachuxe9}{%
\subsection{Caché}\label{cachuxe9}}

La figura muestra un extracto del contenido de un cache de 128 KB de 2
grados de asociativad y líneas de 16 bytes. El computador posee un bus
de direcciones de 20 bits. El cache se organiza en 2 bancos, cada uno
con 4096 líneas. Por ejemplo en la línea 4f2 (en hexadecimal) del banco
izquierdo se almacena la línea de memoria que tiene como etiqueta 04f2
(es decir, la línea que va de la dirección 04f20 en hexadecimal a la
dirección 04f2f)

\begin{longtable}[]{@{}lllll@{}}
\toprule()
& Banco & 1 & Banco & 2 \\
\midrule()
\endhead
línea & etiqueta & contenido & etiqueta & contenido \\
301 & 4301 & & 2301 & \\
4f2 & 04f2 & & a4f2 & \\
c36 & dc36 & & 1c36 & \\
\bottomrule()
\end{longtable}

\hypertarget{arquitecturas}{%
\subsection{Arquitecturas}\label{arquitecturas}}

Considere el siguiente programa en assembly:

\begin{verbatim}
a. lw x7, 4(x5)
b. add x9, x2, x3
c. blt x0, x7, p
d. slli x9, x3, 1
...
p. ori x5, x9, 7
q. srli x1, x5, 2
r. and x10, x7, x9
s. xori x8, -1, x1
\end{verbatim}
