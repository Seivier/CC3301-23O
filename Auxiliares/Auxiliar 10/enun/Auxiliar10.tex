\documentclass[dcc]{fcfmcourse}
\usepackage{teoria}
\usepackage{listings}
\usepackage{xcolor}
\usepackage{longtable}

\definecolor{codegreen}{rgb}{0,0.6,0}
\definecolor{codegray}{rgb}{0.5,0.5,0.5}
\definecolor{codepurple}{rgb}{0.58,0,0.82}
\definecolor{backcolour}{rgb}{0.95,0.95,0.92}

\lstdefinestyle{mystyle}{
    backgroundcolor=\color{backcolour},   
    commentstyle=\color{codegreen},
    keywordstyle=\color{magenta},
    numberstyle=\tiny\color{codegray},
    stringstyle=\color{codepurple},
    basicstyle=\ttfamily\footnotesize,
    breakatwhitespace=false,         
    breaklines=true,                 
    captionpos=b,                    
    keepspaces=true,                 
    numbers=left,                    
    numbersep=5pt,                  
    showspaces=false,                
    showstringspaces=false,
    showtabs=false,                  
    tabsize=2
}

\lstset{style=mystyle}
\usepackage[utf8]{inputenc}

\title[10]{Directorios Unix}
\course[CC3301]{Programación de Software de Sistemas}
\professor{Luis Mateu}
\assistant{Vicente González}
\assistant{Iván Henríquez}
% \assistant{Blaz Korecic}
% Si pasas el comando usedate a la clase, la fecha aparecerá bajo la lista de auxiliares.
% Puedes usar el formato de fecha por defecto de latex (y traducirla usando babel)
% o puedes escribir lo que quieras con el comando \date.
% \date{1 de Septiembre, 2015}


\begin{document}
\maketitle

\begin{problems}

\problem \textbf{Búsqueda}\\
Hace poco fue la tarea 3 de PSS, y usted la hizo a altas horas de la noche. Una vez que la
terminó, se fue a dormir, y al día siguiente no se acuerda de en qué carpeta dejó la tarea.
Naturalmente, usted decide crear un programa para encontrarla, que recorra recursivamente
todas las carpetas (también conocidas como directorios) hasta encontrarla. El programa debe
imprimir todos los archivos que encuentre que tengan el nombre “t3.c”, sin incluir carpetas
que tengan ese nombre. No debe seguir links simbólicos. Para hacer el recorrido utilice la
función \texttt{chdir}.

\problem \textbf{Permisos}\\
Haga un programa llamado \texttt{cualquierusr.c} que busque en un directorio si hay archivos para
los que todos los usuarios tengan permisos de lectura y escritura. Si los encuentra, deber´ıa
imprimirlos en pantalla.


\end{problems}
\end{document}
