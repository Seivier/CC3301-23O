\documentclass[dcc]{fcfmcourse}
\usepackage{teoria}
\usepackage{minted}
\usepackage{xcolor}
\usepackage{framed}
\usepackage[utf8]{inputenc}
\definecolor{LightGray}{gray}{0.95}

\title[6]{Archivos}
\course[CC3301]{Programación de Software de Sistemas}
\professor{Luis Mateu}
\assistant{Vicente González}
\assistant{Iván Henríquez}
% \assistant{Blaz Korecic}
% Si pasas el comando usedate a la clase, la fecha aparecerá bajo la lista de auxiliares.
% Puedes usar el formato de fecha por defecto de latex (y traducirla usando babel)
% o puedes escribir lo que quieras con el comando \date.
% \date{1 de Septiembre, 2015}

\begin{document}
\maketitle

\begin{problems}
\problem \textbf{(P1 C2 Otoño 2016)}
Un archivo contiene un diccionario en el siguiente formato:

\begin{center}
\begin{minipage}{0.8\textwidth}
\begin{framed}
  casa:edificación construida para ser habitada:\\
  lluvia:condensación del vapor de agua contenida en las nubes:\\
  embarcación:todo tipo de artilugio capaz de navegar sobre o bajo el agua:\\
  alimento:sustancia ingerida por un ser vivo:\\
  ...etc...
\end{framed}
\end{minipage}
\end{center}

El primer caracter ``:'' separa la palabra de su definición. El segundo ``:'' termina la definición. Todas las líneas del archivo contienen un número fijo de caracteres para que sea sencillo hacer acceso directo con \texttt{fseek}. Las palabras están desordenadas en el archivo. Este archivo no cabe en la memoria del computador. Programe la siguiente función:

\begin{center}
  \texttt{void modificar(char *nom\_dic, char *palabra, char *def, int n\_lin, int ancho);}
\end{center}

Esta función cambia la definición de \texttt{palabra} por \texttt{def} en el diccionario almacenado en el archivo \texttt{nom\_dic}. El parámetro \texttt{n\_lin} es el número de líneas del archivo (y por lo tanto el número de palabras y definiciones) y \texttt{ancho} es el número de caracteres de cada línea en el archivo.

\textbf{Bonus:} Ahora las líneas tienen un largo variable, y ya no está el ``:'' del final de cada línea. Tampoco se conoce la cantidad de líneas del archivo, y se agrega la condición de que si no se encuentra la palabra, entonces se debe agregar al final. Programe la función:

\begin{center}
  \texttt{void modificar(char *nom\_dic, char *palabra, char *def);}
\end{center}

\end{problems}
\end{document}
